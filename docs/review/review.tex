\section{Žaidimo apžvalga}

Žaidime pateikiama tankų šaudyklė, kurios veiksmas vyksta mieste.
Kovos priežastis nėra aiški, bet panašu, jog norima atkurti Antrojo pasaulinio karo vaizdą.

Žaidimo meniu pasitinka geram žaidimui nuteikiančiais elektrinės gitaros rifais. 
Yra galimybė perskaityti pagalbą, kūrėjų sąrašą. Iš esmės - atitinka lūkesčius.

\subsection{Žemėlapis}

Pradėjus žaidimą, žaidėjas su tanku yra išmetamas vidury miestelio.
Kitų tankų tuo metu netoliese nėra ir žaidėjas yra priverstas aiškintis ką daryti toliau.

Žemėlapyje yra daug įvairių pastatų. 
Taip pat keletas medžių, kurie būdami netoliese žaidimą padaro beveik nežaidžiamą.

Smulkmena, bet pastebima: mažai kur išlaikyti masteliai.
Tanko dydis lyginant su aplinka varijuoja nuo žaislinio iki gigantiško.

\subsection{Tankas}


Tankas - neišsiskiriantis, primena amerikietišką "Sherman" tanką.
Šalia įprastų kontrolės būdų turi ir sukiojamą bokštelį su pabūklu.

Tanko ir bokštelio valdymas WASD bei QE klavišais - standartinis ir lengvai perprantamas neskaičius pagalbos, tačiau problemų sukelia kamera.
Sukiojant bokštelį sunku suprasti kur jis yra nutaikytas. 
Tą išspręsti galima keliais būdais:
	\begin{itemize}
		\item Kamerą susiejus ne su tanku, o su bokšteliu. T.y. kamera suktųsi sukant bokštelį, o ne tanką.
		\item Pridėjus taikiklį, kuris leistų suprasti, kur nutaikytas pabūklas.
		\item Įgyvendinus abu paminėtus punktus.
	\end{itemize}

Kitas tanko valdomas elementas - pabūklas. 
Juo šaudoma naudojant "Space" klavišą.
Tai sukelia problemų, jei žaidėjas yra pratęs žaidimus valdyti dvejomis rankomis, 
nors ir nereikalauja atitraukti rankų nuo kitų valdymo mygtukų.
Tai galima būtų pagerinti paskirsčius valdymą 2 rankoms, pvz.:
	\begin{itemize}
		\item Kaip senose šaudyklėse ("Doom") - valdyti tanką rodyklėmis, šaudyti su Ctrl.
		\item Perkelti pabūklo ir bokštelio valdymą pelei.
	\end{itemize}
	
Techninę pusę vertinti, turint galvoje, kad tai yra prototipas - sunku.
Tačiau vairuojant tanką šios problemos išlenda labai erzinančiai.
	\begin{itemize}
		\item Tankas jaučiasi lyg labai lengvas - lengvai apvirsta.
		\item Apvirtęs tankas toliau yra valdomas ir net gali šaudyti bei važinėti (to sprendimas būtų - apvirtusį tanką laikyti "mirusiu")
		\item Tanko šaudymo nei matosi, nei girdisi
	\end{itemize}
		
\subsection{Kiti techniniai trūkumai}
	Žaidimas, nors ir prototipas, turi rimtų techninių problemų.
	Tai ypač pasireiškia veikimo greityje - tam tikrose žemėlapio vietose žaidimas pradeda tiek strigti, kad tampa nebežaidžiamas.
	
	Kitos problemos mažesnės, bet vis tiek pasireiškia nuolat:
	\begin{itemize}
		\item Kiaurai pervažiuojamos sienos.
		\item Šokinėjanti kameros pozicija.
		\item Kamera prasmenganti pastatuose.
	\end{itemize}
	
\subsection{Išvados}
	Žinoma, žaidimą, kaip prototipą vertinti yra sunku, tačiau jis neįtraukia ir nesukelia noro žaisti.
	Tiesa, pagrinde dėl to kaltos techninės problemos, kurias galutiniame variante tikriausiai būtų įmanoma ištaisyti.
	
	Gera pusė - žaidimo apipavidalinimas, meniu muzika, skirtingų pastatų kiekiai.
	Visa tai atrodo visai neblogai ir matosi į tai įdėtas darbas.
	