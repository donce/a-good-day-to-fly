\section{Darbų tvarkaraštis}

\newcommand{\schedule}[1]{
    \tabulinesep = 3mm
    \taburowcolors{white .. gray!20}
    \begin{longtabu}{ X[c, m] X[c, m] X[c, m] }
        \textbf{Laikas} & \textbf{Veikla} & \textbf{Atsakingas narys} \\
        \endhead
        \tabucline{-}
        #1
    \end{longtabu}
    \tabureset
}

\newcommand{\scheduleEntry}[3]{
    #1 & #2 & #3 \\
}

\schedule{
    \scheduleEntry
    {Darbo pradžia -- spalio 6 d.}
    {Dokumentacijos rašymas, prototipo kūrimas.}
    {Visi}
    \scheduleEntry
    {Spalio 6 d. -- lapkričio 3 d.}
    {Prototipo kūrimas pristatymui.}
    {Visi}
    \scheduleEntry
    {Lapkričio 3 d.}
    {Pirmojo prototipo atsiskaitymas.}
    {Visi}
    \scheduleEntry
    {Lapkričio 3 d. -- gruodžio 1 d.}
    {Prototipo užbaigimas.}
    {Visi}
    \scheduleEntry
    {Gruodžio 1 d.}
    {Baigto žaidimo prototipo atsiskaitymas.}
    {Visi}
}
