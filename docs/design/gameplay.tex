\section{Žaidimo mechanika}
\subsection{Žaidimo aplinka}

Miestas sudarytas iš įvairių aukštų, daugiakampio formos pastatų (t.y. tiek stačiakampiai, tiek sudėtingesnių formų).
Kai kuriuose didesniuose pastatuose būtų tuneliai bei ertmės pro kurias galima būtų sutrumpinti kelią (nors ir rizikuojant savo rezultatu).

Trasa sudaryta iš:
\begin{enumerate}
\item Kontrolės punktų, per kuriuos laivas turi praskristi.
\item Kontrolės ruožų, per kuriuos laivas turi praskristi tam tikra kryptimi.
\end{enumerate}
Aplink trasą skrendama tik vieną kartą.

\subsection{Žaidėjo valdomas objektas}

Žaidėjas pilotuos itin greitą futuristinę skraidyklę, panašią į kosminį laivą.
Jos valdyme ir fizikoje nebus siekiama visiško realizmo.
Gravitacija skraidyklės neveiks arba veiks minimaliai.
Skrendant pakeliui bus įmanoma pačiupti laikinus patobulinimus, suteikiančius daugiau galios.

Žaidimo kūrimo metu bus siekiama žaidimą padaryti pakankamai sunkų, kad jis galėtų būti elektroninio sporto šaka.
Skraidyklės valdymas bus sudėtingas, tačiau bus vengiama dirbtinių būdų jam apsunkinti.
%% Pavyzdžiui, galėtų būt horizontalūs ir vertikalūs strafe'ai. Laivas galės pasisukt visom 3 ašim (taigi, ir skrist aukštyn kojom).
Skraidyklės valdymas reikalaus fizinių sugebėjimų -- gero reakcijos laiko ir erdvės suvokimo.
%% Kitaip tariant, greičiai bus vos vos suvokiami amfetamino nepririjusiam gamer'iui.
Valdymo sudėtingumas turėtų žaidimui suteikti ilgą (bet nebūtinai stačią) mokymosi kreivę, kaip kad yra daugelyje elektroninio sporto šakų.