\section{Žaidimo mechanika}
\subsection{Žaidimo aplinka}

Lenktynės vyksta futuristiniame mieste, tarp pastatų.
Miestas ir trasa generuojami atsitiktinai.

Miestas sudarytas iš įvairių aukštų, daugiakampio formos pastatų (t.y. tiek stačiakampiai, tiek sudėtingesnių formų).
Kai kuriuose didesniuose pastatuose yra tuneliai bei ertmės, pro kurias galima būtų sutrumpinti kelią (nors ir rizikuojant savo rezultatu).

Pastatai labai aukšti, jų viršūnės virš debesų.
Pakilti virš jų lenktynių dalyviui sunku.

\subsection{Žaidimo taisyklės}

Žaidimas skirtas keliems žaidėjams.
Žaidėjas gali žaisti ir vienas, tačiau tada siužetas netenka prasmės.

Žaidėjai turi ribotą kiekį laiko, kuriam pasibaigus jie priversti baigti lenktynes.
Mieste išdėstyti kontrolės punktai, kuriuos pasiekęs, žaidėjo laikas pratęsiamas.
Žaidimo tikslas -- paskutiniam išlikti lenktynėse.
Vieno žaidėjo atveju, tikslas -- išlikti kuo ilgiau.

Kontrolės punktai yra vienkartiniai, t.y., laiko jie prideda tik vienam žaidėjui, o po to išnyksta.
Kontrolės punktui išnykus, kitoje miesto dalyje atsiranda naujas kontrolės punktas.
Bendras mieste esančių kontrolės punktų skaičius vienu metu yra pastovus.
Kontrolės punktai yra nedideliame aukštyje, kad lenktynių dalyviams nekiltų pagunda sutrumpinti kelią pakylant virš pastatų.

\subsection{Žaidėjo valdomas objektas}

Žaidėjas pilotuoja itin greitą futuristinę skraidyklę, panašią į kosminį laivą.
Jos valdyme ir fizikoje nėra siekiama visiško realizmo.
Skraidyklė gali staiga, tačiau nedideliu greičiu, pajudėti statmenai savo judėjimo trajektorijai (t.y., atlikti "strafe").
Ji gali pasisukti visomis 3 ašimis, taigi, ir skristi aukštyn kojomis.
Gravitacija skraidyklės neveikia arba veikia minimaliai.

Stiprus smūgis į kliūtį (pastatą ar kitą žaidėją) žaidėjui mirtinas.
Po jo žaidėjas nebegali tęsti lenktynių.
Silpnas smūgis tik sulėtina žaidėją.

Žaidimo kūrimo metu bus siekiama žaidimą padaryti pakankamai sunkų, kad jis galėtų būti elektroninio sporto šaka.
Skraidyklės valdymas bus sudėtingas, tačiau bus vengiama dirbtinių būdų jam apsunkinti.
Skraidyklės valdymas reikalaus fizinių sugebėjimų -- gero reakcijos laiko ir erdvės suvokimo.
%% Kitaip tariant, greičiai bus vos vos suvokiami amfetamino nepririjusiam gamer'iui.
Valdymo sudėtingumas turėtų žaidimui suteikti ilgą (bet nebūtinai stačią) mokymosi kreivę, kaip kad yra daugelyje elektroninio sporto šakų.

\subsection{Modifikatoriai}

Skrendant pakeliui bus įmanoma pačiupti laikinus modifikatorius, pakeičiančius skraidyklės judėjimo dėsnius.
Kai kurie modifikatoriai žalingi.
Modifikatorių rūšys:
\begin{description}
\item[Pagreitinimas]
    Padidina maksimalų ir minimalų greičius.
\item[Sulėtinimas]
    Sumažina maksimalų ir minimalų greičius.
\item[Apsukimas]
    Nukreipia skraidyklę priešinga kryptimi, nei ji judėjo prieš tai.
\item[Stabdžių išjungimas]
    Neleidžia sulėtinti skraidyklės.
    Žinoma, smūgis į kliūtį išlieka efektyvus.
\item[Manevringumo padidinimas]
    Smarkiai padidina skraidyklės manevringumą.
\end{description}
