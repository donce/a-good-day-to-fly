\section{Interfeisas}
%% Taip, tokį žodį naudot galima. http://terminai.vlkk.lt

\subsection{Mini-žemėlapis}

Aktyvūs kontrolės punktai ir kiti objektai žaidėjui rodomi mini-žemėlapyje, esančiame ekrano kampe.

Mini-žemėlapis apvalus.
Jis rodo ribotą atstumą nuo žaidėjo, kuris yra jo centre.
Objektai jame žymimi taškais.
Nutolę už mini-žemėlapio matymo ribų, objektai lieka jo pakraštyje.
Kadangi skristi galima tiek aukštyn kojomis, tiek šonu, žemėlapis atvaizduojamas pagal žemės plokštumą ir, veikėjui judant, nesisuka.

\subsection{Stulpeliai-matuokliai}

Ekrano kampe rodomi įvairių modifikatorių matuokliai.
Jie rodo, kiek laiko dar veiks vienas ar kitas modifikatorius.

\subsection{Ekrano vaizdai}

Pavyzdinis ekrano vaizdas pateiktas \referToPicture{action-shot}

\insertPicture{action-shot}{Pavyzdinis ekrano vaizdas}
