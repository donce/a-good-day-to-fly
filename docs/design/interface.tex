\section{Interfeisas}
%% Taip, tokį žodį naudot galima. http://terminai.vlkk.lt

Žaidėjas iš karto nematys visos trasos; ji atskleidžiama lenktynių bėgyje.
Sekantis trasos kontrolės punktas ar ruožas žaidėjui rodomas mini-žemėlapyje.

\subsection{Mini-žemėlapis}

Mini žemėlapis apvalus, storu rėmeliu, rodys ribotą atstumą nuo žaidėjo.
Objektai žymimi taškais.
Nutolę už mini-žemėlapio matymo ribų, objektai liks jo pakraštyje.
Kadangi skristi bus galima tiek aukštyn kojomis, tiek šonu, žemėlapis bus atvaizduojamas pagal žemės plokštumą ir, veikėjui judant, nesisukios.

\subsection{Stulpeliai-matuokliai}

Ekrano šonuose bus rodomi įvairių patobulinimų ("power-up") matuokliai.
Jie rodys, kiek liko vieno ar kito patobulinimo.

\subsection{Ekrano vaizdai}

Pavyzdinis ekrano vaizdas pateiktas \referToPicture{action-shot}

\insertPicture{action-shot}{Pavyzdinis ekrano vaizdas}
