\section{Techninė apžvalga}

Žaidimas bus kuriamas naudojant Unity žaidimų variklį.

\subsection{Miesto generatorius}

Miestas bus modeliuojamas kaip kvadratinis tinklelis.
Miesto generavimui bus paruoštas rinkinys polyomino\footnote{\url{http://en.wikipedia.org/wiki/Polyomino}} formos pastatų.
Generatorius šį tinklelį padengs atsitiktinai parinktais pastatais iš pastatų rinkinio.

Detalus miesto generavimo algoritmo aprašymas bus pateiktas žaidimo projekto ir dizaino dokumentacijoje.

%% Detalus miesto generavimo algoritmo aprašymas:
%% \begin{enumerate}
%% \item Sudaromas kvadratinio tinklelio pavidalo grafas $G = (V, E)$.
%% \item Visos $E$ priklausančios briaunos, išskyrus kraštines, įtraukiamos į aibę $S$.
%% \item \label{itm:selection}
%%     Iš aibės $S$ atsitiktinai parenkama briauna $ab$.
%% \item \label{itm:removal}
%%     Nagrinėjama, kas atsitiktų, pašalinus iš $E$ briauną $ab$.
%%     Po briaunos pašalinimo grafe atsirastų naujas veidas $f$.
%%     Tikrinama, ar turimame pastatų rinkinyje yra bent vienas $f$ formos pastatas.
%%     Jei taip, briauna $ab$ pašalinama iš aibės $E$.
%%     Jei pašalinus briauną $ab$, viršūnės $a$ arba viršūnės $b$ valentingumas tampa lygus $1$, atitinkama viršūnė ir likusi jai incidenti briauna šalinama iš $G$.
%%     Bet kuriuo atveju $e$ pašalinama iš aibės $S$ ir toliau nebenagrinėjama.
%% \item Žingsniai \ref{itm:selection}--\ref{itm:removal} kartojami tol, kol aibėje $S$ nebelieka briaunų.
%% \item Kiekvienam grafo veidui atsitiktinai parenkamas pastatas, kurio forma sutampa su veido forma.
%% \item Trasa sudaroma iš atsitiktinai parinktų $V$ priklausančių viršūnių.
%% \end{enumerate}
%%
%% Jei pastatų rinkinyje bus bent vienas $1 \times 1$ pastatas, algoritmas tikrai duos teisingą rezultatą.
%%
%% Algoritmo ribotumai:
%% \begin{itemize}
%% \item Jei pastatas nėra paprastai sujungtas\footnote{\url{http://en.wikipedia.org/wiki/Simply_connected_space}}, jame būtinai turi būti ertmė, leidžianti pasiekti jo viduje esančius grafo mazgus.
%% \item Jei pastate yra akligatvio pavidalo ertmė, jos viduje esantis mazgas negali būti įtrauktas į trasą (jis bus pašalintas iš $V$).
%% \end{itemize}
%% Paprasčiausias būdas išspręsti šias problemas būtų uždrausti tokius pastatus, tačiau abi problemas galima išspręsti kartu su pastato forma saugant jo "pralaidumo" informaciją.
