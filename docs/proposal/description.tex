\section{Žaidimo idėja}
\subsection{Vieno sakinio aprašymas}

Kosminių laivų lenktynės futuristiniame mieste.

\subsection{Detalus aprašymas}

%% Žaidimo idėja aprašyta taip, kaip ją įsivaizduoja KD. Skaitykit kaip knygą, o ne pavienes nuotrupas.

%% TODO: gal suskirstyt į subsubsection?

Lenktynės vyksta futuristiniame mieste, tarp pastatų.
Miestas ir trasa generuojami atsitiktinai.

\placeholder{žaidimo tikslas; galbūt žaidėjo pasirodymą galima įvertinti kokiais nors taškais?}

Miestas sudarytas iš aukštų, stačiakampio formos pastatų.
%% Atkreipkit dėmesį: ne vien kvadratinių. Kai kurie galėtų būt sudėtingesnių formų: L, T.
%% Štai ką turiu galvoje, sakydamas "pastatas" (tik gal ne tokios formos): http://goo.gl/frMjYg.
%% Kai kurie pastatai galėtų turėti tunelius viduje, pavyzdžiui: http://goo.gl/r6eJhi. Žinoma, tai prasminga tik tada, jei pastatas pakankamai platus.
Jie mieste išdėstomi atsitiktinai.

Trasa generuojama atsitiktinai.
%% Miestą įsivaizduoju kaip kvadratinio tinklelio pavidalo grafą. Pastatai užima vieną ar daugiau veidų (tokiu atveju kai kurių briaunų trūksta).
Ji sudaryta iš:
\begin{enumerate}
\item Kontrolės punktų, per kuriuos laivas turi praskristi.
      %% Kitaip tariant, grafo mazgas.
\item Kontrolės ruožų, per kuriuos laivas turi praskristi tam tikra kryptimi.
      %% Kitaip tariant, grafo briauna arba iš kelių briaunų sudarytas takas su kryptimi.
\end{enumerate}
Aplink trasą skrendama tik vieną kartą.

Žaidėjas iš karto nematys visos trasos; ji atskleidžiama lenktynių bėgyje.
%% Kraštutiniu atveju vienu metu galėtų būti rodomas tik vienas kontrolės punktas/ruožas. Žaidėjui jį pasiekus, parodomas kitas.
Trasa žaidėjui rodoma mini-žemėlapyje.

Žaidėjas pilotuos kosminį laivą.
Laivo valdyme ir fizikoje nebus siekiama visiško realizmo.
Valdymas bus sudėtingas, bet ne dirbtinai apsunkintas.
%% Pavyzdžiui, galėtų būt horizontalūs ir vertikalūs strafe'ai. Laivas galės pasisukt visom 3 ašim (taigi, ir skrist aukštyn kojom).
Jo sudėtingumas turėtų žaidimui suteikti ilgą (bet nebūtinai stačią) mokymosi kreivę, kaip kad yra daugelyje elektroninio sporto šakų.
Laivas nebus per daug manevringas.
%% Kad nebūtų per lengva. Negerai būtų, jei žaidėjas galėtų skrisdamas visu greičiu staiga apsisukt 180 laipsnių kampu.

Žaidimas bus pakankamai sunkus ir sudėtingas, kad galėtų būti elektroninio sporto šaka.
Jis reikalaus fizinių sugebėjimų -- gero reakcijos laiko ir erdvės suvokimo.
%% Kitaip tariant, greičiai bus vos vos suvokiami amfetamino nepririjusiam gamer'iui.

Žaidime dominuos šviesios spalvos.
Šešėliai bus neintensyvūs.
Nebus ryškios šviesotamsos.
Ant vieno objekto bus matomi tik keli vienos spalvos tonai.
%% Kažkas panašaus į Team Fortress 2, I guess.
Bus vengiama spalvų gausos.
